\documentclass[12pt]{article}                                                                                                                       
\usepackage{sbc-template}                                                 
\usepackage{graphicx,url}                                                 
\usepackage[utf8]{inputenc}                                               
\usepackage[brazil]{babel}                                                      
\usepackage{graphicx}

\title{Cheatsheet -- Teórico\\ Teoria de Grafos e Computabilidade}
\author{Iyan Lucas Duarte Marques\inst{1}}

\address{Instituto de Ciências Exatas e Informática - Pontifícea Universidade Católica Minas Gerais (PUC-MG)}

\begin{document}

\maketitle

\section{O que é o Grafos}
Grafos é um arcabouço matemático utilizado na definição e/ou resolução dos problemas.
Uma forma de abstração matemática para representação de problemas.

\subsection{Conceitos}
\begin{itemize}
	\item \textbf{Grafos:} Coleção de vértices e arestas.
	\item \textbf{Vértices:\footnote{
			      É a bolinha do grafo
		      }} Objeto simples que pode ter nomes e outros atributos.
	      \item\textbf{Aresta:\footnote{
			      É a linha que liga duas bolinhas
		      }} Conexão entre dois vértices.
\end{itemize}

A representação matemática de um grafo se dá pela seguinte forma:

Um grafo $G = (V,E)$ em que $V$ é o conjunto de vértices e $E$ o conjunto de arestas de forma que:
\begin{equation}
	E = \{(u, v) | u, v \in V\}
\end{equation}\footnote{
	Tradução: O conjunto de arestas $E$ é igual à aresta de $u$ a $v$, tal que $u$ e $v$ pertencem ao grupo de vértices (são vértices).
}
Para um grafo direcionado.

\begin{equation}
	E = \{\{u, v\} | u, v \in V\}
\end{equation}\footnote{
	Um grafo não direcionado usa chaves (como se fosse um dicionário/lista) enquanto um direcionado usa parênteses (como uma tupla).
}
Para um grafo não-direcionado.

\subsection{Grafo direcionado}\footnote{
	Outra nomenclatura: dirigido; orientado.
}
É um par $G = (V, E)$, em que $V$ é um conjunto finito e $E$ é uma relação binária em $V$.
A representação das arestas em um grafo direcionado é a $\longrightarrow$ setinha.

\subsection{Grafo não-direcionado}
É um par $G = (V, E)$, em que o conjunto de arestas $E$ consiste em pares de vértices não orientados, ou seja, a aresta $(v_j, v_i)$\footnote{
	Nota: a letra $v$ é a representação de vértice e o símbolo $_i$ representa qual vértice.
} e a $(v_i, v_j)$ são a mesma aresta.

\section{Terminologia}

\subsection{Dos Grafos}

\subsubsection{Grafos Simples}
É um grafo que não possui loops ou arestas paralelas.

\subsubsection{Regularidade}
Um grafo no qual todos os vértices possuem o mesmo grau, é chamado de grafo regular.

\subsubsection{Nulicidade}
Um grafo sem nenhuma aresta é chamado de grafo nulo. Todos os vértices em um grafo nulo são vértices isolados.

\subsubsection{Rotulação}
Um grafo $G = (V, A)$ é dito ser rotulado quando cada elemento\footnote{
	Sendo elemento um grafo, aresta ou qualquer coisa presente no conjunto $G$
} estiver associado a um rótulo [nome].

\subsubsection{Valoração (Ponderado)}
Um grafo $G = (V, A)$ é dito ser valorado quando uma aresta, um vértice ou ambos possuem uma  ou mais funções associando-os a um conjunto numérico.
Ou seja, cada elemento pode conter um peso (valor).

\subsubsection{Completo}
Um grafo $G = (V, A)$ é completo se para cada par de vértices existe uma arestas ligando os dois.
Ou seja, todo mundo tem que estar ligado a todo mundo.\\
$\longrightarrow$ \textit{Em um grafo completo, quaisquer dois vértices distintos são adjancentes $(K_n)$}.
$\longrightarrow$ \textit{Seja $K_n$ um grafo completo com $n$ vértices.
	O número de arestas de um grafo completo é:}\\

\begin{equation}
	|E| = \displaystyle\frac{(n - 1) * n}{2}
\end{equation}\footnote{
	$|E| \longrightarrow$ é o valor absoluto do número de arestas. $n \longrightarrow$ é o número de vértices.
}

\subsubsection{Grafo conexo}
Um grafo é conexo quando existe pelo menos um caminho entre todos os pares do vértices.

\subsubsection{Bipartição}
Um grafo é bipartido quando um conjunto $V$ de vértices pode ser separado em 2 subconjuntos em que todas as arestas do grafo ligue um vértice $v_1$ a um $v_2$.\\
Já um grafo pode ser dito bipartido completo quando cada vértice de $v_1$ é conexo a \textbf{todo} vértice de $v_2$.
O número de arestas de um grafo bipartido completo em 2 subconjuntos $n$ e $m$ é:
\begin{equation}
	|E| = n * m
\end{equation}

\subsubsection{Isomorfismo}
Dois grafos são ditos isomorfos se existir uma correspondência:

\begin{itemize}
	\item vértice pra vértice
	\item aresta pra aresta
	\item O mesmo vértice possui uma aresta que incide para o mesmo vértice
	\item mesmo número de componentes
	\item mesmos graus de vértices
\end{itemize}

\subsubsection{Complementaridade}
Seja $G = (V, E)$ um grafo simples dirigido ou não-dirigido. O grafo complementar de G, denotado por $C(G)$ ou $\bar{G}$.\\
Os vértices de $C(G)$ são todos os vértices de $G$.\\
As arestas de $C(G)$ são exatamente as arestas que faltam em $G$.
para formarmos um grafo completo

\subsubsection{Subgrafo}
Mesma coisa de substring

\subsection{Dos Vértices}

\subsubsection{Vértices adjacentes}
É quando dois vértices estão ligados (pontos finais) de uma $\longrightarrow$ \textit{setinha}.

\subsubsection{Graus de vértices}
O grau de cada vértice é representado pela terminologia $d(v)$.
A forma que se calcula o grau depende do tipo do grafo:
\begin{itemize}
	\item \textbf{Grafo não-direcionado:} o grau $d(v)$ é o numero de arestas ($\longrightarrow$ \textit{setinhas}) que incidem em $v$
	\item \textbf{Grafo direcionado:} o grau $d(v)$ é dividido entre grau de entrada $d^+(v)$ e grau de saída $d^-(v)$.
	      O numero de arestas ($\longrightarrow$ \textit{setinhas}) que incidem em $v$ é o grau de entrada.
	      Os que saem são o grau de saida
\end{itemize}

A soma dos graus de todos os vértices de um grafo $G$ é duas vezes o número de arestas de $G$, e portanto é par.
\begin{equation}
	\Sigma_{i=1}^n d(v_i) = 2e
\end{equation}

\textbf{\textit{ATENÇÃO:}} Um loop conta duas vezes pro grau do vértice.

\subsubsection{Incidência}
Quando um vértice $v$ é o vértice final (quando o vértice não aponta pra ninguém, só recebe a $\longrightarrow$ \textit{setinhas}) de alguma aresta $e = uv$\footnote{
	Nota: as letras estão em minúsculo porque estão representando uma aresta e uma vértice, diferente da letra maiúscula que representa o grafo e o conjunto de aretas.
}, é dito que $u$ é incidente [ou incide] em $v$.

\subsubsection{Vértice isolado}
Um vértice com nenhuma aresta incidente é chamado de vértice isolado

\subsubsection{Pendência}
Um vértice com grau 1 é chamado de vértice pendente

\subsection{Das Arestas}

\subsubsection{Loop}
É quando tem uma aresta que vai pro mesmo vértice, tipo uma máquina de estado finito.
A aresta $(v_i, v_i)$

\subsubsection{Arestas Paralelas}
Quando um vértice manda duas $\longrightarrow$ \textit{setinhas} para outro vértice.

\subsubsection{Adjacência}
Duas arestas \textbf{não paralelas} são adjacentes se elas incidem a um vértice comum, ou seja, se duas arestas que não saem do mesmo vértice apontam a $\longrightarrow$ \textit{setinha} pra um mesmo vértice.


\subsection{Das Operações}

\subsubsection{Caminhada \textit{(walk)}}
\begin{itemize}
	\item Uma caminhada num grafo $G = (V, E)$ é uma sequência finita e não nula $W$\footnote{
		      Ex.: $W = v_0 e_1 v_1 e_2 \dots e_k v_k$
	      }
	\item A caminhada precisa alternar entre vértices e arestas, tal que, para $1 \leq i \leq k$, as extremidades da aresta $e_i$ são $v_(i-1)$ e $v_i$.\footnote{
		      Ou seja, na caminhada, não se pode andar de um vértice para outro se eles não estão ligados.
	      }
	\item Nós definimos que $W$ é a caminhada de $v_0$ [a orígem] para $v_k$ [o término]
\end{itemize}

\subsubsection{Trilha \textit{(Trail)}}
\begin{itemize}
	\item Uma trilha é uma caminhada em $G = (V, E)$ se as arestas de $W$ são distintas
	\item Não se pode repetir aresta
\end{itemize}

\subsubsection{Trajeto \textit{(path)}}
\begin{itemize}
	\item Um trajeto é uma trilha em $G = (V, E)$ se os vértices de $W$ são distintos
	\item Não se pode repetir aresta, nem vértices
\end{itemize}

\subsubsection{Ciclo \textit{(cycle)}}
\begin{itemize}
	\item Um ciclo é um trajeto fechado, onde a orígem e o término são a mesma aresta
	\item Não se pode repetir aresta, nem vértices
\end{itemize}

\subsubsection{União}
Seja $G_1=(V_1, A_1)$ e $G_2 = (V_2, A_2)$ dois grafos, o grafo $G = G_1 \cup G_2$ representa a união de dois grafos.
Isso significa que ambos os conjuntos de vértices e arestas são unidos, \textbf{mas} não se cria novas arestas para se conectarem.
Podem ser aplicadas a qualquer número finito de grafos e são operações associativas e comutativas.

\subsubsection{Soma}
Seja $G_1=(V_1, A_1)$ e $G_2 = (V_2, A_2)$ dois grafos, o grafo $G = G_1 + G_2$ representa a soma de dois grafos.
Isso significa que ambos os conjuntos de vértices são somados, \textbf{e se cria} novas arestas para conectarem cada vértice de cada subconjunto.
Podem ser aplicadas a qualquer número finito de grafos e são operações associativas e comutativas.

\subsubsection{Remoção}

\begin{itemize}
	\item \textbf{\textit{Remoção de aresta:}}
	      Se $E$ é uma aresta de um grafo $G$, denota-se $G-$ e o grafo obtido de $G$ pela remoção da aresta e.
	      Se $E$ é um conjunto de arestas em $G$, denota-se $G-E$ ao grafo obtido pela remoção das arestas em $E$.
	\item \textbf{\textit{Remoção de vértice:}}
	      Se $v$ é um vértice de um grafo $G$ denota-se por $G - v$ o grafo obtido de $G$ pela remoção do vértice $v$ conjuntamente com as arestas incidentes a $v$.
	      Denota-se $G - S$ ao grafo obtido pela remoção dos vértices em $S$,
	      sendo S um conjunto qualquer de vértices de $G$
	\item \textbf{\textit{Contração de aresta:}} Denota-se por G/e o grafo obtido pela contração da aresta $e$.
	Quando se remove a aresta $e = (v, w)$ de $G$, se une os vértices resultantes de forma que as $\longrightarrow$ \textit{setinhas} dos dois antigos vértices apontem para o novo.
\end{itemize}





\end{document}
